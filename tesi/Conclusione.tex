\chapter*{Conclusione}
\addcontentsline{toc}{chapter}{Conclusione}
\phantomsection
\chaptermark{Conclusione}

In questo elaborato è stato presentato il concetto di Network Slicing e come il suo impatto potrebbe influenzare diverse tipologie di servizi.\\
È stato poi analizzato un potente strumento di simulazione, Mininet, mostrandone alcuni utilizzi di base e spiegandone le affinità con le reti di tipo SDN. Sono stati mostrati anche diversi metodi per analizzare alcune tra le prestazioni che una rete può essere in grado di offrire.\\
Questi concetti e questi strumenti sono stati poi uniti per la creazione di una rete basata sull'idea di Network Slicing, dando la possibilità di osservare come questa tipologia di architettura possa impattare la costruzione, la gestione e soprattutto l'utilizzo di un'infrastruttura.\\\\
L'esempio svolto riguarda una rete piuttosto semplice. Tuttavia, queste tecnologie possono essere usate per progettare, simulare ed infine costruire reti di dimensioni e complessità molto maggiori, andandone ad influenzare molto sensibilmente l'efficienza e le prestazioni. 