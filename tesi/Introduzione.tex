\chapter*{Introduzione}
\addcontentsline{toc}{chapter}{Introduzione}
\phantomsection
\chaptermark{Introduzione}

Negli ultimi anni machine learning e deep learning hanno rivoluzionato il mondo dell'intelligenza artificiale e della computer vision ottenendo risultati impressionanti in molti compiti come classificazione di immagini, segmentazione semantica ecc. superando, in alcuni casi, anche i risultati ottenuti da un utente umano \cite{imagenet}.\\
Per usare tutti questi strumenti abbiamo bisogno di una grande quantità di dati al fine di supervisionare il training di reti neurali. Per questa ragione sono stati creati una grande quantità di dataset contenenti immagini e le rispettive annotazioni come ad esempio KITTI \cite{KITTI}, Matterport3D \cite{Matter} e Replica \cite{Replica}.\\\
La fase di annotazione, come si può facilmente immaginare, è un processo lento, noioso e molto propenso ad errori. Labellare una singola immagine in 2D può richiedere alcune ore ed è un compito che richiede anche una certa esperienza con l'utilizzo di alcuni software. La difficoltà poi cresce se decido di lavorare con dati 3D come mesh o pointcloud.\\\\
Per semplificare e velocizzare questa parte si è deciso di realizzare \textit{Shooting Labels} un tool per la segmentazione semantica 3D basato su realtà virtuale. L'utente viene trasportato all'interno di una scena rappresentata come mesh 3D, dove tutte le superfici possono essere "colorate" semanticamente. L'esperienza immersiva fornita dalle tecnologie VR consente all' utente di muoversi fisicamente all'interno dello scenario etichettare e interagire con gli oggetti in modo naturale e coinvolgente rendendo questo compito, oltre che più veloce, anche divertente come giocare a un video-game \cite{SL}.\\\\
In questo elaborato si parlerà dapprima della segmentazione semantica e di tutti i problemi ad essa connessi, verranno poi presentati gli strumenti utilizzati nel corso della stesura della tesi e tirocinio per poi passare ad una trattazione dettagliata del software. Infine verranno analizzati e commentati alcuni dati ottenuti dall'utilizzo di questo strumento.